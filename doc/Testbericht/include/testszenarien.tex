%Beschreiben, ob die Tests erflogreich durchführbar sind. Einleitungstext überarbeiten.
Die Testszenarien orientieren sich an den in Kapitel Usecases im Entwurfsdokument definierten Abläufen. Die Testszenarien wurden von einem Entwickler durchgeführt.
\begin{enumerate}[align=left, leftmargin=4em, label={\textbf{\textbackslash T4.\arabic*\textbackslash}} ]
\newcommand{\testsz}[1]{
	\item \textbf{\usecaseTitle{#1}}\\
	\usecase{#1}
	\textbf{Nachbedingung:}\\
}

\testsz{1} Im Feld "{}Images" erscheint eine Vorschau der gewählten Bilder.
\result{1}
\testsz{2} Die Ansicht in der GUI repräsentiert die ausgewählten Algorithmen.
\result{1}
\testsz{3} Die GUI repräsentiert die geänderten Einstellungen und die Algorithmen übernehmen die Änderungen. Letzteres ist nur prüfbar, wenn die Änderungen erhebliche Auswirkungen in der Visualisierung nach sich ziehen.
\result{1} Die Änderung der Parameter wurde mit dem Debugger verifiziert anstatt visuell über die Ausgabe der Algorithmen.
\testsz{4} Es wird eine Datei an dem gewählten Ort angelegt und nach dem Laden der Einstellungen
werden Algorithmen und Einstellungen so übernommen, wie sie abgespeichert wurden.
\result{2} Die Settings werden implizit durch die Ausführung mit den Ergebnissen gespeichert. Laden ist nicht implementiert.
\testsz{5} Nach dem Laden der Datei sind die Algorithmen und Einstellungen so gesetzt, wie sie beim Abspeichern waren.
\result{4}
\testsz{6} Der Benutzer sieht ein Fenster, in dem er Anleitung und Hilfe zu dem Programm und seiner Benutzung findet.
\result{4}
\testsz{7} Die Nachrichten im Logfenster sind nur noch von den ausgewählten Schweregraden.
\result{1}
\testsz{8} Die Ausführung der Algorithmen wurde erfolgreich beendet und die Ergebnisse können angesehen und manipuliert werden. Es wird in dem eingestellten Verzeichnis eine Ordnerstruktur mit den Zwischenergebnissen angelegt.
\result{1}
\testsz{9} Es liegen Anzeigedaten bis zum letzten ausgeführten Algorithmus vor. Das eingestellte Verzeichnis enthält Zwischenergebnisse bis zum letzten ausgeführten Algorithmus.
\result{1}
\testsz{10} Nach Anwahl einer Ansicht werden die entsprechenden Ergebnisse dort Visualisiert.
\result{1}
\end{enumerate}