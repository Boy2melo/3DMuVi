\newcommand{\usecaseTitle}[1]{
	\IfEqCase{#1}{
		{1}{Laden von Einzelbildern}
		{2}{Auswählen von Algorithmen}
		{3}{Ändern von Einstellungen}
		{4}{Speichern der Einstellungen}
		{5}{Laden von Einstellungen}
		{6}{Öffnen der Hilfe}
		{7}{Filtern der Logs}
		{8}{Ausführen von Algorithmen}
		{9}{Unterbrechung der Verarbeitung}
		{10}{Vorschau der Ergebnisse}
	}
}

\newcommand{\usecase}[1]{
	\IfEqCase{#1}{
		{1}{
			\textbf{Vorbedingung:}\\
			Das Programm wurde gestartet
			\\\textbf{Ablauf:}
			\begin{enumerate}
				\item Auswählen der "{}Lade Bilder" im Menü
				\item Suchen und Öffnen eines oder mehrerer Bilder
			\end{enumerate}
		}
		{2}{
			\textbf{Vorbedingung:}\\
			Das Programm wurde gestartet
			\\\textbf{Ablauf:}
			\begin{enumerate}
				\item Auswahl der Algorithmen für die einzelnen Schritte im rechten Bereich der
				GUI
			\end{enumerate}
		}
		{3}{\textbf{Vorbedingung:}\\
			Es wurden für die einzelnen Schritte entsprechende Algorithmen ausgewählt.
			\\\textbf{Ablauf:}
			\begin{enumerate}
				\item Im linken Bereich der GUI können die einzelnen Einstellungen für die Algorithmen geändert werden
			\end{enumerate}
		}
		{4}{
			\textbf{Vorbedingung:}\\
			Es wurden Algorithmen für die einzelnen Schritte ausgewählt
			\\\textbf{Ablauf:}
			\begin{enumerate}
				\item Der Benutzer wählt im Menü die Option "{}Speichern"
				\item Es wird eine Zieldatei gewählt, in die das Setup gespeichert werden soll
				\item Durch Klick auf "{}Speichern" werden die selektierten Algorithmen und deren Einstellungen gespeichert.
			\end{enumerate}
		}
		{5}{
			\textbf{Vorbedingung:}\\
			Das Programm wurde gestartet
			\\\textbf{Ablauf:}
			\begin{enumerate}
				\item Der Benutzer wählt im Menü die Option "{}Laden"
				\item Im folgenden Dialog sucht der Nutzer eine passende Datei
				\item Durch Klick auf "{}Öffnen" werden die Algorithmen und ihre Einstellungen geladen
			\end{enumerate}
		}
		{6}{
			\textbf{Vorbedingung:}\\
			Das Programm wurde gestartet
			\\\textbf{Ablauf:}
			\begin{enumerate}
				\item Der Benutzer wählt im Menü "{}Hilfe" aus
			\end{enumerate}
		}
		{7}{
			\textbf{Vorbedingung:}\\
			Das Logfenster zeigt bereits ein paar Nachrichten an
			\\\textbf{Ablauf:}
			\begin{enumerate}
				\item Der Benutzer wählt ein oder mehrere Loglevel aus
			\end{enumerate}
		}
		{8}{
			\textbf{Vorbedingung:}\\
			Es wurden Algorithmen und Bilder ausgewählt und geladen.
			\\\textbf{Ablauf:}
			\begin{enumerate}
				\item Der Benutzer wählt "{}Start"
				\item Die Abarbeitung der Algorithmen auf den Eingabedaten beginnt nun
			\end{enumerate}
		}
		{9}{
			\textbf{Vorbedingung:}\\
			Die Verarbeitung wurde erfolgreich gestartet
			\\\textbf{Ablauf:}
			\begin{enumerate}
				\item Durch die Anwahl von "{}Stop" wird das weiterreichen der Daten angehalten
				\item Nach dem Abschluss des aktuell laufenden Algorithmus kommt die Abarbeitung zum Stillstand
			\end{enumerate}
		}
		{10}{
			\textbf{Vorbedingung:}\\
			Die Algorithmen wurden erfolgreich ausgeführt
			\\\textbf{Ablauf:}
			\begin{enumerate}
				\item Durch Auswählen der einzelnen Vorschaumodi in der Mitte der GUI wählt der Benutzer die Daten, die er sehen möchte.
			\end{enumerate}
		}
	}	
}