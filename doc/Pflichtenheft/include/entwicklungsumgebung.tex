% Zur Entwicklung verwendete Hard‐ und Software. In der Pflichtenheft‐Phase soll  sich  das  Team  in  die  Werkzeuge  einarbeiten  und  sich  hier  vorläufig  festlegen.  Zu  diesen Werkzeugen zählen unter anderem ein UML‐Modellierungswerkzeug, IDE, Code‐Verwaltungssystem.

\section{Allgemein}
	\begin{itemize}
		\item \LaTeX\ zur Erstellung von Dokumenten
		\item TeXstudio zum Schreiben der \LaTeX-Dokumente (Teammitglieder können, falls gewünscht, davon abweichen, da die \LaTeX-Datei unabhängig von dem Editor ist.)
		\item Git als Versionskontrolle
		\item Der aktuelle Entwicklungsstand ist auf github abrufbar
		\newline(\href{https://github.com/boitumeloruf/3DMuVi}{https://github.com/boitumeloruf/3DMuVi})
		\item Das Fraunhofer-Institut stellt uns einen FTP-Server bereit, auf dem wir nicht-öffentliche Dateien austauschen können.
	\end{itemize}
\section{Implementierung}
	Anwendungen zur Umsetzung des Projektes:
	\begin{itemize}
		\item Die Anwendung wird in der Sprache C++ umgesetzt. Wenn sinnvoll, sind Funktionen aus C++11 und C++14 zu verwenden.
		\item Als Compiler wird Clang verwendet um die Entwicklung durch bessere Fehlermeldungen des Compilers zu erleichtern.
		\item Als integrierte Entwicklungsumgebung ist Qt Creator zu verwenden. Auch darin integrierte Tools, wie bspw. Qt Designer zur grafischen Erstellung von Oberflächen, sollen verwendet werden.
	\end{itemize}
\section{Hilfsbibliotheken}
	Hilfsbibliotheken dienen der einfacheren Implementierung durch die Nutzung von bereits durch Dritte implementierte Funktionen.
	\begin{itemize}
		\item Qt soll zur Umsetzung der grafischen Benutzeroberfläche verwendet werden. Dabei soll Qt Widgets verwendet werden, um eine native Darstellung der Oberfläche zu erreichen. Außerdem kann Qt für Funktionen genutzt werden, die dort aber nicht in C++ enthalten sind.
		\item PCL enthält Funktionen zur 3D Rekonstruktion von Bildern und Anzeige der Ergebnisse. Außerdem sind in PCL bereits Datenstrukturen definiert wie sie für den Austausch von Daten mit Plug-Ins benötigt werden.
	\end{itemize}
\section{Validierung}
	Tools zur Überprüfung auf vorhersehbare Fehler:
	\begin{itemize}
		\item Qt Test für automatisierte Tests von einzelnen Einheiten der Anwendung.
	\end{itemize}
\section{Erstellung von Grafiken}
	Anwendungen um Grafiken erstellen zu können werden individuell von dem erstellenden Teammitglied gewählt. Einzige Anforderung dabei ist, dass es eine Datei exportieren kann, die in \LaTeX\ eingebunden werden kann.
