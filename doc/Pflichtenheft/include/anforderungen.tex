
\section{Musskriterien}
\subsection{Modul: Workflow}
Die Software soll einen standardisierten 4-Phasen-Workflow bereitstellen, bestehend aus:

\begin{enumerate}
		\item Feature Extraktion / Matching
		\item Posenschätzung
		\item Tiefenschätzung
		\item 3D Fusion
\end{enumerate}
Für jede Phase dieses Workflows sollen verschiedene Algorithmen ausgewählt werden können.
%\begin{itemize}
%	\item Die Software soll einen standardisierten 4-Phase-Workflow bereitstellen, bestehend aus:
%	\begin{enumerate}
%		\item Feature Extraktion / Matching
%		\item Posenschätzung
%		\item Tiefenschätzung
%		\item 3D Fusion
%	\end{enumerate}
%	\item Für jede Phase dieses Workflows sollen verschiedene Algorithmen ausgewählt werden können.
%\end{itemize}
\subsection{Modul: Input/Output}
Das Programm soll eine Menge von Einzelbildern einlesen und diese an die Algorithmen zur Bearbeitung übergeben. Außerdem muss es möglich sein, die Ergebnisse bzw. Zwischenergebnisse abzuspeichern.
%\begin{itemize}
%	\item Das Programm soll eine Menge von Einzelbildern einlesen und diese an die Algorithmen zur Bearbeitung übergeben können.
%	\item Ergebnisse bzw. Zwischenergebnisse sollen gespeichert werden können.
%\end{itemize}
\subsection{Modul: Visualisierung}
Nach dem Ausführen des Workflows sind die Ergebnisse auf verschiedene Arten darstellbar. So lässt sich zum Beispiel das berechnete 3D Modell als Point Cloud visualisieren. In dieses Modell sollen Kameraposen und -orientierungen als Kamerapyramide eingezeichnet werden können. Des Weiteren soll eine Tiefenkarte der Ergebnisse als Visualisierung zur Verfügung stehen. Zuletzt soll es eine Ansicht geben, in der die berechneten Features, die durch die Algorithmen erkannt wurden, in Einzelbildern eingezeichnet sind. 
%Darstellung der Ergebnisse:
%\begin{itemize}
%	\item Features die durch die Algorithmen erkannt wurden, sollen in Einzelbildern eingezeichnet werden.
%	\item Kameraposen und -orientierungen sollen im 3D Modell als Kamerapyramide eingezeichnet werden .
%	\item Anzeigen der Tiefenkarten
%	\item Anzeige des 3D Modells als Piont Cloud.
%\end{itemize}
\subsection{Modul: Logger}
Informationen, Warnungen, Fehler und Debugmeldungen, die von den Algorithmen ausgeworfen werden, soll das Programm aufzeichnen, anzeigen und abspeichern können.
\subsection{Modul: Einstellungen}
Die Einstellungen der einzelnen Verfahren können abgespeichert und geladen werden.
\subsection{Modul: Interaktion}
Der Workflow soll gestartet und gestoppt werden können.\\Das Stoppen der Verarbeitung bewirkt, dass keine Daten weitergegeben werden und lediglich die restliche Berechnung der Workflowstufe ausgeführt wird. Zudem soll ein globaler Arbeitsindikator die andauernde Verarbeitung der Verfahren anzeigen.
%\begin{itemize}
%	\item Der Workflow soll gestartet und gestoppt werden können.\\Das Stoppen der Verarbeitung bewirkt, dass keine Daten weitergegeben werden und lediglich die restliche Berechnung der Workflowstufe ausgeführt wird.
%	\item Ein globaler Arbeitsindikator soll die andauernde Verarbeitung der Verfahren anzeigen.
%\end{itemize}

\section{Kannkriterien}
\subsection{Modul: Workflow}
Zu dem bereits existierenden 4-Phasen-Workflow sollen weitere fest implementierte Workflows zu Verfügung stehen. Die Konfiguration dieser Workflows soll abgespeichert und geladen werden können. Zusätzlich soll es möglich sein, Workflows dynamisch zu erstellen und zu verändern. 
%\begin{itemize}
%	\item Es sollen mehrere fest implementierte Workflows zu Verfügung stehen.
%	\item Workflows können variable erstellt und verändert werden.
%	\item Abspeichern und laden der Workflow-Konfiguration.
%\end{itemize}
\subsection{Modul: Visualisierung}
Neben der Point Cloud als Darstellungsvariante gibt es die Möglichkeit das 3D Modell als Mesh bzw. texturiert zu visualisieren.
%\begin{itemize}
%	\item Darstellung des 3D Modells in Form von Point Cloud, Mesh und Texturiert.
%\end{itemize}
\subsection{Modul: Einstellungen}
Die globalen Einstellungen des Programms können abgespeichert und geladen werden.
\subsection{Modul: Interaktion}
Beim Programmstart können die Workflow-Konfiguration und das Verzeichnis mit den Eingangsdaten als Commandline-Optionen übergeben werden. Außerdem soll es möglich sein, einzelne Ausführungsstufe des Workflows separat zu startet und die dafür benötigten Daten aus z.B. Ergebnisse und Zwischenergebnisse vorangegangener Berechnungen zu laden. Jede dieser Ausführungsstufen soll einen eigenen Arbeitsindikator erhalten. Zudem soll das Programm die Funktion bieten, Daten neu zu sortieren und den Workflow auf Untergruppen der Daten anzuwenden. Auch das wiederholte Ausführung von einzelnen Workflow-Stufen auf verschiedene Datensätze soll zu Verfügung stehen. Zuletzt soll bei Bedarf eine Interaktion mit den Ergebnisse und Zwischenergebnisse, wie z.B. das Bearbeiten der Kameras, Ein-, Ausblenden und Entfernen einzelner Punkte möglich sein.
%\begin{itemize}
%	\item Auswahl der Workflow-Konfiguration und Verzeichnis mit Eingangsdaten durch Commandline-Optionen beim Programmstart.	
%	\item Laden vorhergegangener (Zwischen-) Ergebnisse.
%	\item Einzelne Schritte sollen gestartet werden können.
%	\item Arbeitsindikatoren für jeden einzelnen Verarbeitungsschritt.
%	\item Daten neu Sortieren / Ausführen auf Untergruppen der Daten.
%	\item Automatisierte wiederholte Ausführung von einzelnen Algorithmen auf verschiedenen Datensätze.
%	\item Manipulation und Interaktion mit (Zwischen-) Ergebnisse.\\(z.B. Ein-, Ausblenden und Entfernen einzelner Punkte, Kameras, Matches etc.).
%\end{itemize}

\section{Abgrenzungskriterien}
\begin{itemize}
	\item Videos liegen als sortierte Einzelbilder vor.
	\item Algorithmen müssen nicht Implementiert werden. (Bereitstellung von Schnittstellen zur Einbindung)
	\item Eine Änderung der Eingabe erfordert eine erneute Ausführung der Algorithmen.
	\item Die Benutzeroberfläche beschränkt sich auf englisch Sprache.
	\item Für die Entwicklung wird ausschließlich Qt 5.5 genutzt.
\end{itemize}

\section{Nichtfunktionalitäten}
		\begin{itemize} 
			\item Entwicklung für Linux, aber möglichst Plattform-unabhängig (Qt)
			\item Responsive Oberfläche während der Ausführung
			\item Läuft auf einzelnen Rechnern
			\item Zielgruppe sind durchschnittliche PC-Benutzer, die sich mit den Algorithmen und den damit verbundenen Workflows auskennen
			\item Folgen der Coding Conventions der Abteilung VID (siehe Anhang)
			\item Alle Namen und Kommentare im Quellcode in Englisch
			\item Quellcode Dokumentation in Doxygen
			\item Methoden und Klassen eine Dokumentation
		\end{itemize}