% Detailliertere  Beschreibung  der  Funktionalität,  wiederum  gegliedert  in Grundfunktionen und optionale Funktionen. 

Die Beschreibung der Funktionalität gliedert sich in Grundfunktionen und optionale Funktionen.

\section{Grundfunktionen}
\section{Funktional}
\subsection{Obligatorische Funktionen}
\textbf{Modul: Workflow}
\begin{enumerate}[ align=left, label={\textbf{\textbackslash FM1\arabic*0\textbackslash}} ]
	\item Standardisierter 4-Phase-Workflow bestehend aus:
\begin{enumerate}[ align=left, label={\textbf{\arabic*}} ]
	\item [Einlesen von Bildern]
	\item Feature Extraktion / Matching
	\item Posenschätzung
	\item Tiefenschätzung
	\item 3D Fusion
	\item [Fertiges Modell]
	\end{enumerate}
	sollte ausgewertet werden.
	\item Auswahl verschiedener Algorithmen für jede Phase
\end{enumerate}
\textbf{Modul: I/O}
\begin{enumerate}[ align=left, label={\textbf{\textbackslash FM2\arabic*0\textbackslash}} ]
	\item Einlesen und Übergabe von Einzelbildern
	\item Abspeicherung der (Zwischen-) Ergebnisse
	\item Speicherort in verschiedenen Verzeichnissen: Ein Hauptverzeichnis, darunter für jede Phase
	ein Unterverzeichnis (Auswählbar)
\end{enumerate}
\textbf{Modul: Visualisierung}
\begin{enumerate}[ align=left, label={\textbf{\textbackslash FM3\arabic*0\textbackslash}} ]
	\item Visualisierung der (Zwischen-) Ergebnisse
	\item Einzeichnen der Features in Einzelbildern
	\item Einzeichnen der Kamerposen und -orientierungen im 3D Modell (Kamerapyramide)
	\item Anzeigen der Tiefenkarten
	\item Anzeigen des 3D-Modells (PCL)
\end{enumerate}
\textbf{Modul: Logger}
\begin{enumerate}[ align=left, label={\textbf{\textbackslash FM4\arabic*0\textbackslash}} ]
	\item Aufzeichnung, Anzeige und Abspeichern eines Logs zur Nachverfolgung von
	\begin{enumerate}[ align=left, label={\textbf{\arabic*}} ]
		\item Allgemeine Informationen
		\item Warnungen
		\item Fehlermeldungen
		\item Debugging
	\end{enumerate}
\end{enumerate}
\textbf{Modul: Einstellung}
\begin{enumerate}[ align=left, label={\textbf{\textbackslash FM5\arabic*0\textbackslash}} ]
	\item Abspeichern und Laden von Einstellungen der einzelnen Verfahren
\end{enumerate}
\textbf{Modul: Interaktion}
\begin{enumerate}[ align=left, label={\textbf{\textbackslash FM6\arabic*0\textbackslash}} ]
	\item Starten und Stoppen der Verarbeitung
	\item Stopp => keine Weitergabe der Daten, aktuelle Verarbeitung läuft zu Ende.
	\item Globaler Arbeitsindikator
\end{enumerate}	

\subsection{Optionale Funktionen}

\textbf{Modul: Visualisierung}
\begin{enumerate}[ align=left, label={\textbf{\textbackslash FK1\arabic*0\textbackslash}} ]
 \item Manipulation und Interaktion mit (Zwischen-) Ergebnissen, wie beispielsweise
Ein-, Ausblenden und Entfernen einzelner Punkte/Kameras/Matches, sollte gegeben sein
 \item Schalter zur Auswahl der Darstellung des 3D-Modells
Point Cloud, Mesh, Texturiert
Auch hier: Daten werden geliefert, lediglich Auswahl welche Daten angezeigt werden
\end{enumerate}
\textbf{Modul: Workflows}
\begin{enumerate}[ align=left, label={\textbf{\textbackslash FK2\arabic*0\textbackslash}} ]
 \item Weitere (feste) Workflows
 \item Variable Workflow-Definition als Plugin
 \item Abspeichern und Laden der Workflow-Konfiguration
\end{enumerate}
\textbf{Modul: Interaktion}
\begin{enumerate}[ align=left, label={\textbf{\textbackslash FK3\arabic*0\textbackslash}} ]
\item Starten einzelner Schritte,
 Laden vorhergegangener (Zwischen-) Ergebnisse
\item Automatisierte, wiederholte Ausführung von einzelnen Algorithmen auf verschiedene Datensätze
\item Auswahl der Workflow-Konfiguration und Verzeichnis mit Eingangsdaten durch Command-Line-Optionen beim Programmstart (versteckte Ausführung)
\item Daten neu Sortieren / Ausführung auf Untergruppe der Daten
\item Arbeitsindikator für jeden Schritt
\end{enumerate}
\textbf{Modul: Einstellung}
\begin{enumerate}[ align=left, label={\textbf{\textbackslash FK4\arabic*0\textbackslash}} ]
\item Abspeichern und Laden von globale Einstellungen
\end{enumerate}

\subsection{Nichtunktionalität}
\begin{enumerate}[ align=left, label={\textbf{\textbackslash NF1\arabic*0\textbackslash}} ]
\item Entwicklung für Linux aber möglichst Plattform-unabhängig (Qt)
\item Responsive Oberfläche während der Ausführung
\item Läuft auf einzelnen Rechnern
\item Zielgruppe sind durchschnittliche PC-Benutzer, die sich mit den Algorithmen und den damit
verbundenen Workflows auskennen
\item Folgen der Coding Conventions der Abteilung VID (siehe Anhang)
\item Alle Namen und Kommentare im Quellcode in Englisch
\item Quellcode Dokumentation in Doxygen
\item Methoden und Klassen eine Dokumentation
\item Gui-Sprache: Englisch
\end{enumerate}


\subsection{Abgrenzungskriterien}
\begin{enumerate}[ align=left, label={\textbf{\textbackslash AK1\arabic*0\textbackslash}}]
\item Videos liegen als sortierte Einzelbilder vor
\item Algorithmen müssen nicht Implementiert werden
(Bereitstellung von Schnittstellen zur Einbindung)
\item Eine Änderung der Eingabe erfordert eine erneute Ausführung der Algorithmen
\end{enumerate}

\subsection{Daten}
\begin{enumerate}[ align=left, label={\textbf{\textbackslash D1\arabic*0\textbackslash}}]
\item Parameter für Algorithmen
\item Parameter der Workflows
\item (Zwischen-) Ergebnisse zum Austausch zwischen den Algorithmen (Datenaustausch)
\item (Zwischen-) Ergebnisse zum Visualisieren und Abspeichern
\end{enumerate}

\subsection{Schnittstellen}
\begin{enumerate}[ align=left, label={\textbf{\textbackslash S1\arabic*0\textbackslash}}]
\item (Funktionale-) Schnittstellen zum Aufrufen und Datenaustausch der Algorithmen
\item  Schnittstelle zum Einstellen der Algorithmen
Angabe der Config Datei und direkte Angabe eines Parameter-Trees
\end{enumerate}