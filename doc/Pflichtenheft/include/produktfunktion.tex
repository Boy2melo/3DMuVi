% Detailliertere  Beschreibung  der  Funktionalität,  wiederum  gegliedert  in Grundfunktionen und optionale Funktionen. 

Die Beschreibung der Funktionalität gliedert sich in Grundfunktionen und optionale Funktionen.

\section{Grundfunktionen}
	\paragraph{Obligatorische Funktionen}
	\subsection{Modul: Workflow}
	\begin{enumerate}[ align=left, label={\textbf{\textbackslash 	FM1\arabic*0\textbackslash}} ]
		\item Konfigurieren des 4-Phase-Workflow bestehend aus
		\begin{enumerate}[  align=left, label={\textbf{\textbackslash FM11\arabic*\textbackslash}} ]
			\item Einlesen von Bildern
			\item Auswahl verschiedener Algorithmen für Feature Extraktion / Matching
			\item Auswahl verschiedener Algorithmen für Posenschätzung
			\item Auswahl verschiedener Algorithmen für Tiefenschätzung
			\item Auswahl verschiedener Algorithmen für 3D Fusion
			\item Ausgabe des berechneten Modells
		\end{enumerate}
		\item Führe den Workflow aus
	\end{enumerate}

	\subsection{Modul: I/O}
		\begin{enumerate}[ align=left, label={\textbf{\textbackslash FM2\arabic*0\textbackslash}} ]
			\item Import der Einzelbildern
			\item Abspeichern des Ergebnisses
		\end{enumerate}

	\subsection{Modul: Visualisierung}
		\begin{enumerate}[ align=left, label={\textbf{\textbackslash FM3\arabic*0\textbackslash}} ]
			\item Visualisierung des Ergebnissess
			\item Anzeige der Features in Einzelbildern
			\item Einzeichnen der Kamerposen und -orientierungen im 3D Modell (Kamerapyramide)
			\item Anzeigen der Tiefenkarten
			\item Anzeigen des 3D-Modells (PCL)
		\end{enumerate}

	\subsection{Modul: Logger}
		\begin{enumerate}[ align=left, label={\textbf{\textbackslash FM4\arabic*0\textbackslash}} ]
			\item Stelle ein Log zur Nachverfolgung von
			\begin{enumerate}[ align=left, label={\textbf{\textbackslash FM41\arabic*\textbackslash}} ]
			\item Allgemeine Informationen
			\item Warnungen
			\item Fehlermeldungen
			\item Debugging
		\end{enumerate}
		bereit
		\item zeige/blende Log an/aus
		\item Log speichern
	\end{enumerate}

	\subsection{Modul: Einstellung}
		\begin{enumerate}[ align=left, label={\textbf{\textbackslash FM5\arabic*0\textbackslash}} ]
			\item speichere die Einstellungen eines Workflows ab
			\item lade die Einstellungen eines Workflows
		\end{enumerate}

	\subsection{Modul: Interaktion}
		\begin{enumerate}[ align=left, label={\textbf{\textbackslash FM6\arabic*0\textbackslash}} ]
			\item Starten der Verarbeitung
			\item Stoppen der Verarbeitung. Keine Weitergabe der Daten, aktuelle Verarbeitung läuft noch zu Ende.
			\item Globaler Arbeitsindikator anzeigen
		\end{enumerate}	

\section{Optionale Funktionen}
	\subsection{Modul: Workflows}
		\begin{enumerate}[ align=left, label={\textbf{\textbackslash FK2\arabic*0\textbackslash}} ]
			 \item Auswahl weiterer (feste) Workflows
			 \item Konfiguriere neue Workflow-Definitionen als Plugin
			 \item Abspeichern der Workflow-Konfiguration
			 \item Laden der Workflow-Konfiguration
		\end{enumerate}
		
	\subsection{Modul: I/O}
		\begin{enumerate}[ align=left, label={\textbf{\textbackslash FK2\arabic*0\textbackslash}} ]
			\item Speichern der Daten in verschiedenen Verzeichnissen: Ein Hauptverzeichnet und je ein neues Verzeichnis für jeden Algorithmus
			\item speichern der Zwischenergebnisse
			\item Laden vorhergegangener (Zwischen-) Ergebnisse
		\end{enumerate}

	\subsection{Modul: Visualisierung}
		\begin{enumerate}[ align=left, label={\textbf{\textbackslash FK1\arabic*0\textbackslash}} ]
			 \item Manipulation mit den (Zwischen-) Ergebnissen
			 \item Anzeigen und Ausblenden der berechneten Daten aus den Algorithmen im 3D-Modell(zum Beispiel Point Cloud, Mesh, Texturiert. Auch hier: Daten werden geliefert, lediglich Auswahl welche Daten angezeigt werden)
		\end{enumerate}

	\subsection{Modul: Interaktion}
		\begin{enumerate}[ align=left, label={\textbf{\textbackslash FK3\arabic*0\textbackslash}} ]
			\item Starte einzelne Algorithmen ohne Workflow
			\item führe mehrere Workflowkonfigurationen oder Algorithmen  auf den gleichen Datenbestand aus
			\item führe auf mehrere Datenbestände eine Workflowkonfiguration oder Algorithmus aus
			\item Auswahl der Workflow-Konfiguration und Verzeichnis mit Eingangsdaten durch Command-Line-Optionen beim Programmstart (versteckte Ausführung)
			\item Datensatz neu Sortieren
			\item Ausführung der Bearbeitung auf Untergruppen der Daten
			\item Arbeitsindikator für jeden Schritt anzeigen
		\end{enumerate}

	\subsection{Modul: Einstellung}
		\begin{enumerate}[ align=left, label={\textbf{\textbackslash FK4\arabic*0\textbackslash}} ]
		\item Abspeichern und Laden von globale Einstellungen
		\end{enumerate}

\section{Nichtunktionalität}
		\begin{enumerate}[ align=left, label={\textbf{\textbackslash NF1\arabic*0\textbackslash}} ]
			\item Entwicklung für Linux, aber möglichst Plattform-unabhängig (Qt)
			\item Responsive Oberfläche während der Ausführung
			\item Läuft auf einzelnen Rechnern
			\item Zielgruppe sind durchschnittliche PC-Benutzer, die sich mit den Algorithmen und den damit verbundenen Workflows auskennen
			\item Folgen der Coding Conventions der Abteilung VID (siehe Anhang)
			\item Alle Namen und Kommentare im Quellcode in Englisch
			\item Quellcode Dokumentation in Doxygen
			\item Methoden und Klassen eine Dokumentation
			\item Gui-Sprache: Englisch
		\end{enumerate}

%\subsection{Daten}
%\begin{enumerate}[ align=left, label={\textbf{\textbackslash D1\arabic*0\textbackslash}}]
%\item Parameter für Algorithmen
%\item Parameter der Workflows
%\item (Zwischen-) Ergebnisse zum Austausch zwischen den Algorithmen (Datenaustausch)
%\item (Zwischen-) Ergebnisse zum Visualisieren und Abspeichern
%\end{enumerate}

%\subsection{Schnittstellen}
%\begin{enumerate}[ align=left, label={\textbf{\textbackslash S1\arabic*0\textbackslash}}]
%\item (Funktionale-) Schnittstellen zum Aufrufen und Datenaustausch der Algorithmen
%\item  Schnittstelle zum Einstellen der Algorithmen
%Angabe der Config Datei und direkte Angabe eines Parameter-Trees
%\end{enumerate}