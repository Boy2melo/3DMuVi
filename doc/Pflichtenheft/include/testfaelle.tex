% Testfälle  für  die  einzelnen  Produktfunktionen,  die  alle  abgedeckt  sein müssen. Testszenarien für typische Anwendungsszenarien.
Die Testfälle und Testszenarien dienen der Überprüfung der Software bezüglich ihrer Funktionalität und Konsistenz. Testfälle unterscheiden sich von den Testszenarien darin, dass diese automatisierte, somit atomare Funktionstests darstellen, während Testszenarien manuelle Testanweisungen sind, die durch eine ungeschulte Testperson durchgeführt werden sollen.

\section{Testfälle}
Die Testfälle sind alle Automatisiert und werden mit der in Qt integrierten Test-Suite durchgeführt.
\subsection{Funktionstests}
Zuerst wird jede Klasse einzeln bearbeitet. Jede Öffentliche Methode wird mit zwei Testfällen getestet. Der erste Testfall testet das Verhalten einer Funktion im normalen Umfeld mit gewöhnlichen Parametern. Dieser Test dient dazu, Fehler in Berechnungsvorschriften und Implementierungen zu finden. Der zweite Testfall ruft die Funktion mit ungewöhnlichen Parametern auf, eventuell auch mit falschen Datentypen. Diese Tests dienen dazu, die Robustheit der Funktionen zu garantieren. Die Argumente, die an eine Funktion übergeben werden, sind so weit wie möglich Zufällig verteilt zu wählen. Für das nachvollziehen von Fehlern werden die getesteten Parameter zusätzlich zu der Fehlerbeschreibung ausgegeben.
\subsection{Komponententests}
Komponententests testen das Verhalten eines isolierten Moduls der Anwendung. Viele der Tests lassen sich allerdings erst nach der Entwurfsphase festlegen, weshalb dieser Abschnitt bis zum Vollständigen Entwurf der Anwendung noch nicht vollständig sein wird.
\\
	\begin{enumerate}[align=left, leftmargin=4em, label={\textbf{\textbackslash T2.\arabic*\textbackslash}} ]
		\item \begin{tabular}{|c|c|}
			\hline Modul & Funktion \\ 
			\hline I\textbackslash O & FM210 \\ 
			\hline 
		\end{tabular}\\ 
		\subitem \textbf{Ziel:}\\ Testen der Funktionalität Laden und Weitergeben von Eingabebildern.
		\subitem \textbf{Dummies:} \begin{itemize}
			\item Algorithmus, der die empfangenen Bilder an den Test meldet.
			\item Interaktion, die das Modul mit vordefinierten Bildern versorgt.
		\end{itemize}
		\subitem \textbf{Erwartetes Verhalten:}\\
		Die vom Algorithmus gemeldeten Bilder entsprechen den Bildern, die dem Modul übergeben wurden.
		\\\item \begin{tabular}{|c|c|}
			\hline Modul & Funktion \\
			\hline I\textbackslash O & FM220 \& FM230 \\
			\hline
		\end{tabular}\\
		\subitem \textbf{Ziel:}\\ Testen der Fähigkeit, Zwischenergebnisse strukturiert abzuspeichern.
		\subitem \textbf{Dummies:} \begin{itemize}
			\item Algorithmen, die Zufällig verteilte Zwischenergebnisse liefern, deren Art und Typ nicht weiter Definiert ist.
		\end{itemize}
		\subitem \textbf{Erwartetes Verhalten:}\\
		Es wird eine Ordnerstruktur in dem vom Test vorgegebenen Verzeichnis angelegt mit den gespeicherten Daten.
		\\\item \begin{tabular}{|c|c|}
			\hline Modul & Funktion \\
			\hline Logger & FM410 \\
			\hline
		\end{tabular}\\
		\subitem \textbf{Ziel:} \\ Testen der Logfunktionalität
		\subitem \textbf{Dummies:} \begin{itemize}
			\item Generator von Lognachrichten
		\end{itemize}
		\subitem \textbf{Erwartetes Verhalten:}\\
		Nach dem Anlegen der Logs und dem auswählen der einzelnen Loglevel werden nur Nachrichten des gewählten Loglevels zurückgegeben.
		\\\item \begin{tabular}{|c|c|}
			\hline Modul & Funktion \\
			\hline Logger & FM410 \\
			\hline
		\end{tabular}\\
		\subitem \textbf{Ziel:} \\ Testen der Speicherfunktionalität
		\subitem \textbf{Dummies:} \begin{itemize}
			\item Generator von Lognachrichten
		\end{itemize}
		\subitem \textbf{Erwartetes Verhalten:}\\
		Nach dem Anlegen der Logs und dem Abspeichern wird eine Datei an dem spezifizierten Pfad vorgefunden.
		\\\item \begin{tabular}{|c|c|}
			\hline Modul & Funktion \\
			\hline Einstellungen & FM510 \\
			\hline
		\end{tabular}\\
		\subitem \textbf{Ziel:} \\ Test der Einstellungen für Algorithmen
		\subitem \textbf{Dummies:} \begin{itemize}
			\item Algorithmus, der eine Liste von Einstellungen bereitstellt, welche alle Datentypen abdeckt und bei Variationsmöglichkeiten der Typen bzw. Wertebereiche diese Stichprobenartig vorhanden sind.
		\end{itemize}
		\subitem \textbf{Erwartetes Verhalten:}\\
		Nach dem Abspeichern findet sich eine Datei an dem spezifizierten Pfad. Nach dem Laden dieser Datei enthält ein neu Instantiierter Algorithmus dieselben Einstellungen wie der erste Algorithmus, welcher zur Speicherung verwendet wurde.
		\\\item \begin{tabular}{|c|c|}
			\hline Modul & Funktion \\
			\hline Einstellungen & FK410 \\
			\hline
		\end{tabular}\\
		\subitem \textbf{Ziel:} \\ Test der Einstellungen für Globale Parameter
		\subitem \textbf{Dummies:} \begin{itemize}
			\item Keine
		\end{itemize}
		\subitem \textbf{Erwartetes Verhalten:}\\
		Nach dem Sichern globaler Einstellungen findet sich eine Datei an dem spezifizierten Pfad. Nach dem Lesen dieser Datei in eine neue Einstellungs-Instanz enthält diese dieselben Daten wie die Instanz, die zur Speicherung verwendet wurde.
		\\\item \begin{tabular}{|c|c|}
			\hline Modul & Funktion \\
			\hline Workflows & FK230 \\
			\hline
		\end{tabular}\\
		\subitem \textbf{Ziel:}\\ Testen des Abspeicherns von Workflow-Konfigurationen
		\subitem \textbf{Dummies:} \begin{itemize}
			\item Keine
		\end{itemize}
		\subitem \textbf{Erwartetes Verhalten:}\\ Nach dem zufälligen Zusammenbau eines Workflows kann dieser in eine Datei gesichert werden und entspricht nach dem laden dem gesicherten Workflow.
		\\\item \begin{tabular}{|c|c|}
			\hline Modul & Funktion \\
			\hline Interaktion & FK310 \\
			\hline
		\end{tabular}\\
		\subitem \textbf{Ziel:}\\ Testen des Ladens von Zwischenergebnissen
		\subitem \textbf{Hinweis:}\\ Wenn die Funktionalität implementiert wird, muss dieser Test mit dem Test \textbackslash T2.20\textbackslash zusammengeführt werden.
		\subitem \textbf{Dummies (Ergänzung):}\begin{itemize}
			\item Ein Algorithmus, der die Ausgabe des ersten Algorithmus als Eingabe nimmt und ihre Validität prüft.
		\end{itemize}
		\subitem \textbf{Erwartetes Verhalten (Ergänzung):}\\ Nach dem Laden der Zwischenergebnisse kann der zweite Algorithmus die Daten erfolgreich Validieren.
		\\\item \begin{tabular}{|c|c|}
			\hline Modul & Funktion \\
			\hline Foo & Bar \\
			\hline
		\end{tabular}\\
		\subitem Platzhalter für weitere Komponententests
	\end{enumerate}
\subsection{Oberflächentests}
Die Oberflächentests lassen sich mit der Qt Test-Suite ebenfalls automatisieren, eventuell muss allerdings bei der Designphase daran gedacht werden, gerade was das Laden von Dateien angeht.\\
\begin{enumerate}[align=left, leftmargin=4em, label={\textbf{\textbackslash T3.\arabic*\textbackslash}} ]
	\item \textbf{Tab Log}
	\subitem\textbf{Dummies:} \begin{itemize}
		\item Nachrichtengenerator, der für jedes Loglevel eine Nachricht in den Log schreibt
	\end{itemize}
	\subitem\textbf{Ablauf:}\\ Der Nachrichtendummy schreibt für jeden Loglevel eine Nachricht in den Log. Danach wird jeder Loglevel einzeln ausgewählt und die Textbox auf die erwartete Lognachricht überprüft.
	\\\item \textbf{Tab Images}
	\subitem \textbf{Ablauf:}\\ Der Test lädt über die GUI Bilder in das Programm und prüft dann im Tab Images, ob die erwartete Anzahl an Vorschaubildern vorhanden ist.
	\subitem \textbf{Hinweis:}\\
	Das Menü "{}Load Images" kann nicht direkt getestet werden, da ein Dateidialog geöffnet wird. Stadtessen übergeht der Test den Dialog und ruft direkt die Lade-Funktion auf. Die GUI-Klasse muss entsprechend geplant werden.
	\\\item \textbf{Algorithmenwahl}
	\subitem \textbf{Dummies:}\begin{itemize}
		\item Algorithmus, der als auffindbarer Platzhalter dient
	\end{itemize}
	\subitem \textbf{Ablauf:}\\ Es wird für jeden Schritt ein Algorithmus aus der ComboBox ausgewählt. Danach wird im Backend überprüft, ob der Algorithmus auch an die Kontrollschicht weitergegeben wurde und ob es der richtige ist.
	\\\item \textbf{Ausführung der Algorithmen}
	\subitem \textbf{Dummies:}\begin{itemize}
		\item Ein Algorithmus für jeden Verarbeitungsschritt, der etwa zwei Sekunden wartet. Daten sind egal.
	\end{itemize}
	\subitem \textbf{Ablauf:}\\ Im Backend werden die Dummy-Algorithmen aktiviert. In der GUI wird dann der Button "{}Start" gedrückt. Nun müssen alle Interaktiven Elemente, die Einfluss auf die Verarbeitung nehmen, deaktiviert sein. Bekannt sind die Einstellungen für den Algorithmus, die Bilderliste und die Auswahl der Algorithmen. Als nächstes wird geprüft, ob der Arbeitsindikator des ersten und zweiten Algorithmus funktionieren. Ist die Ausführung beim dritten Algorithmus angelangt, wird geprüft, ob die ersten beiden Schritte ihren Zustand in "{}Finished" geändert haben. Zudem wird die Ausführung angehalten. Sobald auch der dritte Algorithmus den Zustand "{}Finished" eingenommen hat, wird geprüft, ob der vierte Algorithmus wie gewünscht nicht gestartet wurde. Zuletzt wird noch geprüft, ob die Oberfläche wieder Interaktiv geschaltet wurde.
	\\\item \textbf{Einstellungen}
	\subitem \textbf{Dummies:}\begin{itemize}
		\item Ein Algorithmus mit verschiedenen Einstellungen und Validierungen, die dem Test bekannt sind.
	\end{itemize}
	\subitem \textbf{Ablauf:}\\ Der Algorithmus wird im Backend aktiviert. Der Test setzt nun in der Oberfläche Werte für die verschiedenen Einstellungen. Es wird geprüft, ob die Werte in den Algorithmus übernommen werden. Danach werden Werte gewählt, die zu Validierungsfehlern führen. Es wird geprüft, ob die Werte im Algorithmus fälschlicherweise geändert wurden. Zudem wird das entsprechende Feld auf eine Visuelle Signalisierung des Fehlers geprüft.
\end{enumerate}
\section{Testszenarien}
Die Testszenarien orientieren sich an den in Kapitel \ref{ch:usecases}: \nameref{ch:usecases} definierten Abläufen. Die Testszenarien sind von einem Benutzer durchzuführen, der idealerweise wenig bis gar keine Kenntnis über das Programm hat.
\begin{enumerate}[align=left, leftmargin=4em, label={\textbf{\textbackslash T4.\arabic*\textbackslash}} ]
\newcommand{\testsz}[1]{
	\item \textbf{\usecaseTitle{#1}}\\
	\usecase{#1}
	\textbf{Nachbedingung:}\\
}

\testsz{1} Im Feld "{}Images" erscheint eine Vorschau der gewählten Bilder.
\testsz{2} Die Ansicht in der GUI repräsentiert die ausgewählten Algorithmen.
\testsz{3} Die GUI repräsentiert die geänderten Einstellungen und die Algorithmen übernehmen die Änderungen. Letzteres ist nur prüfbar, wenn die Änderungen erhebliche Auswirkungen in der Visualisierung nach sich ziehen.
\testsz{4} Es wird eine Datei an dem gewählten Ort angelegt und nach dem Laden der Einstellungen
werden Algorithmen und Einstellungen so übernommen, wie sie abgespeichert wurden.
\testsz{5} Nach dem Laden der Datei sind die Algorithmen und Einstellungen so gesetzt, wie sie beim Abspeichern waren.
\testsz{6} Der Benutzer sieht ein Fenster, in dem er Anleitung und Hilfe zu dem Programm und seiner Benutzung findet.
\testsz{7} Die Nachrichten im Logfenster sind nur noch von den ausgewählten Schweregraden.
\testsz{8} Die Ausführung der Algorithmen wurde erfolgreich beendet und die Ergebnisse können angesehen und manipuliert werden. Es wird in dem eingestellten Verzeichnis eine Ordnerstruktur mit den Zwischenergebnissen angelegt.
\testsz{9} Es liegen Anzeigedaten bis zum letzten ausgeführten Algorithmus vor. Das eingestellte Verzeichnis enthält Zwischenergebnisse bis zum letzten ausgeführten Algorithmus.
\testsz{10} Nach Anwahl einer Ansicht werden die entsprechenden Ergebnisse dort Visualisiert.
\end{enumerate}