% Beschreibung der Funktionalität der zu entwickelnden Systemkomponente. 
% o Musskriterien: Mindestanforderungen. 
% o Kannkriterien: Zusätzliche Funktionalität. 
% o Abgrenzungskriterien: Was gehört nicht zum Funktionsumfang? (vgl. auch Produkteinsatz) 
Die Zielbestimmung definiert die Kriterien des Programms. Zum einen die Musskriterien, also die Funktionen die das Programm mindestens bereit stellen muss und zum anderen die Kannkriterien also mit welchen Funktionen das Programm sinnvoll erweitert werden könnte. Zusätzlich wird in den Abgrenzungskriterien explizit ausgeschlossen was das Programm nicht leisten muss. Die Muss- und Kannkriterien untergliedern sich jeweils in sechs logische Module des Programms. Bei diesen Modulen handelt es sich um:
\begin{enumerate}
	\item \textbf{Workflow} Abstrahiert die Anordnung der Algorithmen.
	\item \textbf{Input/Output} Eingabe, Übergabe, Speichern und Laden von Daten.
	\item \textbf{Visualisierung} Art der Darstellung der Ergebnisse.
	\item \textbf{Logger} Aufzeichnen von relevanten Ereignissen und Meldungen der Algorithmen.
	\item \textbf{Einstellungen} Konfigurationsmöglichkeiten der Software.
	\item \textbf{Interaktion} Bedienung und Rückmeldung des Programms.
\end{enumerate}