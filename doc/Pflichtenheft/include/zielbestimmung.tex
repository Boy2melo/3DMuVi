% Beschreibung der Funktionalität der zu entwickelnden Systemkomponente. 
% o Musskriterien: Mindestanforderungen. 
% o Kannkriterien: Zusätzliche Funktionalität. 
% o Abgrenzungskriterien: Was gehört nicht zum Funktionsumfang? (vgl. auch Produkteinsatz) 
Die Zielbestimmung definiert die Kriterien des Programms. Zum einen die Musskriterien, also die Funktionen die das Programm mindestens bereit stellen muss und zum anderen die Kannkriterien also mit welchen Funktionen das Programm sinnvoll erweitert werden könnte. Zusätzlich wird in den Abgrenzungskriterien explizit ausgeschlossen was das Programm nicht leisten muss. Die Muss- und Kannkriterien untergliedern sich jeweils in sechs logische Module des Programms. Bei diesen Modulen handelt es sich um:
\begin{enumerate}
	\item \textbf{Workflow} Abstrahiert die Anordnung der Algorithmen.
	\item \textbf{Input/Output} Eingabe, Übergabe, Speichern und Laden von Daten.
	\item \textbf{Visualisierung} Art der Darstellung der Ergebnisse.
	\item \textbf{Logger} Aufzeichnen von relevanten Ereignissen und Meldungen der Algorithmen.
	\item \textbf{Einstellungen} Konfigurationsmöglichkeiten der Software.
	\item \textbf{Interaktion} Bedienung und Rückmeldung des Programms.
\end{enumerate}

\section{Abgrenzungskriterien}
Die Benutzeroberfläche wird nur auf englischer Sprache umgesetzt.
Das Ziel dieses Projektes ist die Entwicklung eines Frameworks. Dieses Framework soll in der Lage sein, vorderfinierte Workflows mit ausgewählten Algorithmen zu verwenden, um eine 3D Konstruktion zu erstellen. \textit{Die Algorithmen selbst sind nicht Teil des Projekts und werden bereitgestellt. Auch die Workflows werden festdeefiniert vorgegeben.} Die notwendigen Schnittstellen für die Workflow und Algorithmen sind sicherzustellen.
Falls eine Workflowkonfiguration gewählt wurde und die Berechnung gestartet ist, ist es nicht mehr möglich, Veränderungen in den Workflowkonfiigurationen vorzunehmen. Alle Einstellungen sind \textit{vor} Beginn der Berechnung feszulegen. Beim vorzeitigen, manuellen Abbruch wird nicht sichergestellt, dass die bis zu den Punkt berechneten Daten verwendbar sind. 