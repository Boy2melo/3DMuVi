% Beschreibung der Funktionalität der zu entwickelnden Systemkomponente. 
% o Musskriterien: Mindestanforderungen. 
% o Kannkriterien: Zusätzliche Funktionalität. 
% o Abgrenzungskriterien: Was gehört nicht zum Funktionsumfang? (vgl. auch Produkteinsatz) 

\section{Problemabgrenzung}
Aus einer Menge von Bildern eines Objektes wie z.B. eines Gebäudes soll eine digitale 3D-Rekonstruktion berechnet werden.\newline
Da es keinen einzelnen Algorithmus gibt, der für dieses Problem eine gute Lösung bietet werden Algorithmen der Feature Extraktion / Matching, der Posenschätzung, der Tiefenschätzung und 3D Fusion kombiniert und parametrisiert. Die Kombinationen der Algorithmen ergeben einen mehrstufigen Worfklow. Wenn nun eine solche Kombination von Algorithmen getestet wird, ist es erforderlich die Eingangsdaten und Parameter pro Algorithmus manuell einzustellen, um sie auszuführen. Anschließend muss mit spezieller Software das entstandene 3D-Modell visualisiert werden, um die Ergebnisse validieren zu können.\newline
Diese Umstände resultieren in einem wenig intuitiven und effektiven Umfeld zur Testung von Workflows.

\section{Lösungsansatz}
Um diesen zähen Prozess zu beschleunigen, soll ein Framework entwickelt werden, das den Großteil der Schritte automatisiert, wodurch der Anwender sich lediglich um die Komposition und Konfiguration der Algorithmen kümmern muss. Dazu sollen zunächst Bilder geladen werden können, die anschließend mit dem Start des Workflows an diesen übergeben werden. Dieser Workflow ist in Verarbeitungsschritte unterteilt, die als Plugins geladen und anschließend schnell ausgetauscht und konfiguriert werden können. Die daraus entstanden Workflow-Konfiugrationen können gespeichert und geladen werden, um z.B. ein Testen des Workflows auf eine anderen Bildermenge zu einem späteren Zeitpunkt zu erleichtern. Abschließend sollen die Ergebnisse im Programm visualisiert werden.