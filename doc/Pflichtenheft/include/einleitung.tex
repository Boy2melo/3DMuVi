% vollständige Beschreibung der Aufgabenstellung.
Die Welt wird digital - und mit ihr viele Prozesse. Die Kommunikation läuft heutzutage größtenteils mit digitalen Hilfsmitteln. Auch Geschäftsprozesse werden immer mehr Digital abgewickelt, soweit möglich. Um dies zu bewerkstelligen, müssen die Daten aus der Realen Welt allerdings erstmal digitalisiert werden. Viel ist in dieser Richtung geschehen, wie z.B. Kameras oder Scanner. Doch das einlesen komplexer Daten ist bis heute ein nicht-triviales Problem.

Dieses Projekt beschäftigt sich besonders mit dem Einlesen von Bildern und der Rekonstruktion von Dreidimensionalen Strukturen aus diesen Bildern. Viele Arbeitsbereiche erfordern eine Speicherung und Übertragung von Daten, wie sie im echten Leben vorzufinden sind - z.B. Dreidimensionale Bilder. In der Medizin werden sie benutzt, um Knochen und Skelette platzsparend verfügbar zu haben, in der Automobilindustrie benötigt man sie, um Prototypen schon vor dem ersten tatsächlichen Fertigungsschritt vor Augen zu haben. Besonders die Spielindustrie steigert ihre Anforderungen nach immer besseren und realistischeren Modellen.

Die für den Prozess verwendeten Algorithmen sind jedoch sehr spezialisiert und decken meist nur einen bestimmten Anwendungsfall ab. Zudem lassen sich für die einzelnen Schritte auch unterschiedliche Algorithmen kombinieren, was die Zusammenstellung für ein perfektes Ergebnis umso schwerer macht.

Ziel des 3D-MuVi Projekts ist es, die Kombination von Algorithmen und das damit verbundene Suchen nach dem bestmöglichen Ergebnis zu vereinfachen. Der Benutzer bekommt die Möglichkeit, verschiedene Algorithmen für die einzelnen Schritte auszuwählen und sich die Ergebnisse anzusehen, um somit das bestmögliche Resultat zu erzielen.