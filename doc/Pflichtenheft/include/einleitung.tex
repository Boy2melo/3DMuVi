% vollständige Beschreibung der Aufgabenstellung.
Die Welt wird digital - und mit ihr viele Prozesse. Die Kommunikation läuft heutzutage größtenteils mit digitalen Hilfsmitteln. Auch Geschäftsprozesse werden immer mehr Digital abgewickelt, soweit möglich. Um dies zu bewerkstelligen, müssen die Daten aus der Realen Welt allerdings erstmal digitalisiert werden. Viel ist in dieser Richtung geschehen, wie z.B. Kameras, die Digitalisierung ganzer Bibliotheken und das erfassen von Echtzeitdaten über Sensornetze sind nur ein Bruchteil. Doch das einlesen komplexer Daten ist bis heute ein nicht-triviales Problem.

Viele Arbeitsbereiche erfordern eine Speicherung und Übertragung von Daten, wie sie im echten Leben vorzufinden sind - z.B. Dreidimensionale Bilder. In der Medizin werden sie benutzt, um Knochen und Skelette platzsparend verfügbar zu haben, in der Automobilindustrie benötigt man sie, um Prototypen schon vor dem ersten tatsächlichen Fertigungsschritt vor Augen zu haben. Besonders die Spielindustrie steigert ihre Anforderungen nach immer besseren und realistischeren Modellen.

Für viele dieser Anforderungen gibt es eine einfach erscheinende Lösung. Man nehme ein reales Objekt und digitalisiere seinen dreidimensionalen Aufbau. Doch leider ist das Problem nicht so einfach, wie es sich zuerst anhört. Es gibt viele verschiedene Verfahren um diese Aufgabenstellung und jedes einzelne bewährt sich besonders in einem spezifischen Problemfeld. Ein Beispiel ist Motion Capture, das besonders für die Rekonstruktion von Bewegungen im Dreidimensionalen Raum über die Zeit genutzt wird, Laserscanner können schnell und zuverlässig Personen vermessen. Dieses Projekt beschäftigt sich gezielt mit der Rekonstruktion von digitalen Repräsentationen der Objekte über zweidimensionale Aufnahmen des gewünschten Gegenstandes oder einer kompletten Szene.

Die für den Prozess verwendeten Algorithmen sind jedoch sehr spezialisiert und decken meist nur einen bestimmten Anwendungsfall ab. Zudem lassen sich für die einzelnen Schritte auch unterschiedliche Algorithmen kombinieren, was die Zusammenstellung für ein perfektes Ergebnis umso schwerer macht. So macht es z.B. einen Unterschied, ob man eine Szene aus einem Videoclip rekonstruieren möchte, ob man einzelne Fotos aus verschiedenen Blickrichtungen hat oder ob es sich um eine Luftaufnahme handelt.

Ziel des 3D-MuVi Projekts ist es, die Kombination von Algorithmen und das damit verbundene Suchen nach dem bestmöglichen Ergebnis zu vereinfachen. Der Benutzer bekommt die Möglichkeit, verschiedene Algorithmen für die einzelnen Schritte auszuwählen und sich die Ergebnisse anzusehen, um somit das bestmögliche Resultat zu erzielen.