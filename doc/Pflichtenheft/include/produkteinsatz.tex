% Beschreibt  Einsatzgebiete,  Zielgruppe  und  Betriebsbedingungen  sowie  notwendige Hard‐ und Software.
%Der Produkteinsatz unterteilt sich in Einsatzgebiete, Zielgruppe und Betriebsbedingungen. Das Einsatzgebiet erläutert für welche Anwendung das Programm gedacht bzw. geeignet ist. An wen sich die Software richtet wird in der Zielgruppe beschrieben.
\section{Einsatzgebiete}
Die Software dient zum Konfigurieren von Algorithmen, welche anschließend in einem Workflow kombiniert werden. Diese ermitteln aus einer Menge von Bildern ein 3D Modell, das von der Software dargestellt wird. Somit kann die Software in allen Gebieten eingesetzt werden, in denen Algorithmen und deren Konfigurationen auf eine gegebene Problemstellung angewendet und getestet werden sollen. 
\section{Zielgruppe}
Die Software richtet sich an Entwickler von Algorithmen für Feature Extraktion / Matching, Posenschätzung, Tiefenschätzung und 3D Fusion. Diese können mithilfe der Software ihre Algorithmen testen und validieren. Im Allgemeinen kann jeder die Software nutzen, der sich bereits eingehend mit der Thematik befasst hat und ein Objekt mithilfe verschiedener Kombinationen von Algorithmen digitalisieren möchte.
\section{Betriebsbedingungen} 
Die Software soll unter Bürobindungen laufen. Vorgesehen ist der Einsatz auf Linux, kann aber theoretisch auch auf Windows und Mac OS genutzt werden. 
