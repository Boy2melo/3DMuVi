\section{Einleitung}

Ziel des 3D-MuVi Projekts ist es, die Kombination von Algorithmen und das damit
verbundene Suchen nach dem bestmöglichen Ergebnis zu vereinfachen. 
Der Benutzer bekommt die Möglichkeit, verschiedene Algorithmen für die einzelnen Schritte auszuwählen
und sich die Ergebnisse anzusehen, um somit das bestmögliche Resultat zu erzielen.  \linebreak 
 - Zitat: Pflichtenheft
\\\\
Dieses Dokument stellt nun den Softwaretechnischen Entwurf des 3D-MuVi Prgramms nach den Musskriterien da.
Der Entwurf ist in UML Form dargestellt und kann somit leicht von der späteren Implementierung in der Programmiersprache C++ mit QT 5 abweichen. 
Sofern es in der Implementierungsphase zur Umsetzung von Kannkriterien kommt, kann und muss dieser Entwurf stellenweise erweitert oder verändert werden.
\\\\
Es folgt zunächst eine Liste der Kannkriterien nach deren Priorisierung. 
Anschließend wird der Architekturstil der Software vorgestellt.
Darauf Folgt der Entwurf der einzelnen Module mit ihren Paketen und Klassen.
Im Anschluss folgen Sequenzdiagramme und Klassendiagramme des Projekts.


\section{Ergänzung Anforderungen}
Im Folgenden werden die Kannkriterien erweitert und in 2 Prioritätsklassen unterschieden.
Falls im Ramen der Implementierung ein odere mehrere Kannkriterien umgesetzt werden können, so werden zunächst die hoch priorisierten Kannkriterien umgesetzt.

\subsection{Hoch Priorisierte Kannkriterien}
Kannkriterien die möglicherweise umgesetzt werden.

\subsubsection{Modul Workflow}
\begin{itemize}
\item Zu dem bereits existierenden 4-Phasen-Workflow sollen weitere fest implementierte Workflows zu Verfügung stehen.
\item Die Konfiguration dieser Workflows soll abgespeichert und geladen werden können.
\end{itemize}
\subsubsection{Modul I/O}
\begin{itemize}
\item temporäres Arbeitsverzeichnis für Algorithmen zu Verfügung stellen
\end{itemize}
\subsubsection{Modul Visualisierung}
\begin{itemize}
\item Neben der Point Cloud als Darstellungsvariante gibt es die Möglichkeit das 3D Modell als Mesh bzw. texturiert zu visualisieren.
\item Originalbild Tab, markierte Thumbnails für Bilder mit Tiefenkarte
\end{itemize}
\subsubsection{Modul Interaktion}
\begin{itemize}
\item Auswahl der Workflow-Konfiguration und Verzeichnis mit Eingangsdaten durch Commandline-Optionen beim Programmstart.
\item Laden vorhergegangener (Zwischen-) Ergebnisse.
\item Einzelne Schritte sollen gestartet werden können.
\item Arbeitsindikatoren für jeden einzelnen Verarbeitungsschritt.
\item Interaktion mit (Zwischen-) Ergebnisse(z.B. Ein-, Ausblenden einzelner Punkte, Kameras, Matches).
\end{itemize}
\subsubsection{Modul Einstellungen}
\begin{itemize}
\item QJason der verendeten Algorithmen in das Ergebnisverzeichnis 
\item Die globalen Einstellungen des Programms können abgespeichert und geladen werden.
\end{itemize}

\subsection{Niedrig Priorisierte Kannkriterien}
Kannkriterien die vermutlich nicht umgesetzt werden können.

\subsubsection{Modul Workflow}
\begin{itemize}
\item Workflows dynamisch erstellen und verändern.
\item Workflows als eigenständige Plugins
\end{itemize} 
\subsubsection{Modul Interaktion}
\begin{itemize}
\item Daten neu Sortieren / Ausführen auf Untergruppen der Daten.
\item Automatisierte wiederholte Ausführung von einzelnen Algorithmen auf verschiedenen Datensätze.
\item Manipulation(Zwischen-) Ergebnisse(z.B. Entfernen einzelner Punkte, Kameras, Matches).
\end{itemize}
