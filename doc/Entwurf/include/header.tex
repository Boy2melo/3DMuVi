\documentclass[a4paper]{scrreprt}

\usepackage[ngerman]{babel}   % deutsche Silbentrennung

\usepackage[utf8x]{inputenc}
\usepackage[T1]{fontenc}
\usepackage{lmodern}			% Schönere Schriftart		
\usepackage{hyperref}

%\usepackage[ansinew]{inputenc}
%\usepackage[utf8]{inputenc}   % wegen deutschen Umlauten & UTF-8

\usepackage{graphicx}					% Grafikpaket laden
\usepackage{hyperref}					% Links
\hypersetup{									% Link-Formatierung entfernen & pdf-Inforamtionen setzten
	pdfauthor={Grigori Schapoval, Nathanael Schneider, Tim Brodbeck, Stefan Wolf, Jens Manig, Laurenz Thiel},
	pdftitle={3D Reconstruction Framework from Multi-View Images (3D-MuVi)},
	colorlinks,
	citecolor=black,
	filecolor=black,
	linkcolor=black,
	urlcolor=black
}
%\usepackage{microtype}		% font expansion
\usepackage{enumerate}
\usepackage{xstring}
\usepackage{enumitem}    % Layout der Aufzählungs-Items manipulieren
\setlist{nosep}
\usepackage[ngerman]{cleveref}
\title{Praxis der Softwareentwicklung:\\3D Reconstruction Framework from Multi-View Images (3D-MuVi)}
\subtitle{Entwurfsdokument}
\author{Grigori Schapoval\and Nathanael Schneider\and Tim Brodbeck\and Stefan Wolf\and Jens Manig\and Laurenz Thiel}
\date{\includegraphics[width=.3\textwidth]{img/logo.pdf}\\\vspace{3mm}WS 2015/16}


% Ein paar makros (Nathanael)
%% Liste ein paar Methoden auf
\newcommand{\beginMembers}{\\ \textbf{Methoden:} \begin{itemize} }
%% Liste ein paar Slots auf
\newcommand{\beginSlots}{\\ \textbf{Slots:} \begin{itemize} }
%% Liste ein paar einzelne Attribute auf
\newcommand{\beginAttributes}{\\ \textbf{Attribute:} \begin{itemize} }
	
\newcommand{\beginSignals}{\\ \textbf{Signale:} \begin{itemize} }
%% Ein neuer Member. Zuerst Name, dann Argumente + evtl Docs, Dann rückgabe, dann beschreibung
\newcommand{\newMember}[4] {\item \textbf{#1} \\ #4 \\Übergabeparameter: #2 \\ Rückgabetyp: #3}
%% Ein neuer Slot. Zuerst name, dann passendes Signal, dann beschreibung
\newcommand{\newSlot}[3]{\item \textbf{#1} \\ #3 \\Signal: #2}
%% Ein neues Signal. Zuerst name, dann parameter, dann beschreibung
\newcommand{\newSignal}[3]{\item \textbf{#1} \\ #3 \\Übergabeparameter: #2 \\ Rückgabetyp: Signal}
%% Ein neues Attribut. Zuerst Name inkl Typ, dann beschreibung
\newcommand{\newAttribute}[2]{ \item \textbf{#1} \\ #2}
%% Ein neuer Abstrakter Member. Siehe \newMember
\newcommand{\newMemberAbstract}[4]{\item \textbf{\textit{#1}} \\ #4 \\Übergabeparameter: #2 \\ Rückgabetyp: #3}
%% Beendet die Memberdeklaration
\newcommand{\closeMembers}{\end{itemize} }